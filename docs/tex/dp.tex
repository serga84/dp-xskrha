% preambule dokumentu
\documentclass[12pt]{article}
\usepackage[utf8]{inputenc}
\usepackage{dipp2}
\begin{document}
\pagestyle{headings}


% uvodni cast zaverecne prace
\titul{Webová architektura v prostředí vysoké zátěže}{Bc. Jakub Škrha}{Ing. Michael Štencl, Ph.D.}{Brno 2012}
\podekovani{Text poděkování}
\prohlaseni{Text prohlášení}{místo a datum prohlášení}
\abstract{Citace práce v anglickém jazyce}{abstrakt práce v anglickém jazyce}
\abstrakt{Citace práce v českém jazyce}{abstrakt práce v českém jazyce}
\obsah
\cislovat{3}



% Uvod
\kapitola{Úvod}
Něco na téma jak vzniká vysoká zátěž na iternetu, jaký je vliv internetu, roustoucí zájem o internet a stoupající zátěž. Uvidíme, napíšu to jako poslední.


% Nezadouci vlivy
\kapitola{Nežádoucí vlivy a důsledky vysoké zátěže}
Co může způsobit vysoká zátěž, pád serverů, projektů, škoda pro firmy


% Zneuziti
\kapitola{Zneužití vysoké zátěže}
Různé typu útoků DoS, Anonymous


% Tri a vice vrstev
\kapitola{Tří a vícevrstvá architektura webových systémů}
Třívrstvá architektura je začátek, ovšem v praci nestačí. Obrázky a příklady architektur projektů s vysokou zátěží. Lehké představení dalších možných vrstev.


% Aplikacni vrstva
\kapitola{Aplikační vrstva}
Aplikační vrstva je základ. Realizuje a sestavuje odpovědi na požadavky uživatelů.

\sekce{Webová aplikační architektura MVC}
Webové projekty jsou dnes tvořeny na této architektuře. Proč a co to je.

\sekce{Druhy aplikačních vrstev}
Něco o JAVA Spring, C sharp, Pearl, Python a PHP.

\sekce{Webový server Apache}
Popis webového serveru apache, jeho úlohy, proč se používá, proč o něm píšu a proč ho já používám pro svoji práci.

\sekce{Programovací jazyk PHP}
Je to rozšířený jazyk pro svoji jednoduchost, čitelnost, ohebnost a dynamičnost.

\sekce{Ajax a webové služby}
Ajax na straně klienta, webová služba na straně aplikace. Stránky jsou dynamické a skládané z více požadavků.



% Databaze
\kapitola{Dabázová vrstva}
Úloha perzistence dat. Stará se o data, odlehčení aplikace, apod.

\sekce{Optimalizace SQL dotazů}
Proč a jak optimalizovat dotazy

\sekce{Indexace}
Proč jsou dobré indexy

\sekce{Partitioning}
Kdy a jak používat partitioning

\sekce{Replikace}
Něco o replikacích a proč se používají.

\sekce{Druhy databází}
Něco málo o MySQL, Oracle, PostgreSQl, MSSQL.

\sekce{PostgreSQL}
Co je to za databázi a proč jsem si ji vybral pro svoji práci.



% Web cache
\kapitola{Webová cache}
Proč webovou cache, proč se používá a k čemu je dobrá. Urychlení odezvy, odlehčení jiným požadavkům na aplikaci, apod.

\sekce{Druhy cache}
Popsat jednotlivé druhy browser, proxy, reverse proxy a aplikační cache

\sekce{Typy obsahu}
Popsat typy obsahu pro cache. Dynamický vs. statický

\sekce{Proxy cache a cache prohlížeče}
Klientská část, kde a kdo ji instaluje a kdy a jak se používá.

\sekce{Reverzní proxy cache}
Serverová část, kde a kdo ji instaluje a kdy a jak se používá.

\podsekce{Nginx}
Popis NGINX a proč jsem ho vybral.

\sekce{Aplikační a distribuovaná cache}
Popis aplikačních a distribuovaných cache, jaký je jejich smysl a kde je jejich použití

\podsekce{Memcached}
Popis Memcached a proč jsem ji vybral


% Dalsi vrstvy
\kapitola{Další vrstvy aplikace}
K čemu jsou další vrstvy

\sekce{CDN}
Content delivery network obrázky a stream.

\sekce{NoSQL Databáze}
K čemu slouží a kde najdou své uplatnění.

\sekce{Vyhledávání}
Z vyhledávání se také dělá další vrstva.



% Virtualizace
\kapitola{Virtualizace}
Projekty dnes neběží vždy na jednom serveru, ale na více virtualizovaných serverech. Proč tomu tak je.


% Load balancing
\kapitola{Load balancing}
Nevím jestli k této kapitole se vůbec dostanu, uvidíme. Každopádně serverů bývá vždy několik a jak zajišťovat toto rozložení zátěže.


% Cloud computing
\kapitola{Cloud Computing}
Budoucnost projektů, startupů, vše řešeno cloudem. AWS


% Prakticka cast
\kapitola{Praktická část a experimenty}
Úvod do toho, že se budu praktickou částí snažit dosáhnout nasymolování vytížené webové architektury a optimalizovat jednotlivé vrstvy v rámci možností.

% Popis aplikace a nastroju
\sekce{Aplikace a její vrstvy}
Představení aplikace, její síťové schéma, jednotlivé vrstvy s popisem, domény, apod.

\sekce{Testovací nástroje}
Popis toho co sleduji testovacími nástroji

\podsekce{XHProf}
K profilování

\podsekce{Siege}
Pro generování zátěže

\podsekce{PostgreSQL Explain}
Vysvětlení sql dotazů

% APC
\sekce{Aplikační vrstva PHP}
Tady se budeme snažit optimalizovat PHP kód.

\podsekce{Optimalizace PHP pomocí APC}
Jak se chovala aplikace bez APC a co dosahnu APC

\podsekce{Dosažené výsledky}
Toto bude vždy na konci každého experimentu, grafy, časy, screeny, apod.

% Databaze
\sekce{Databázová vrstvy}
Tady se budeme snažit optimalizovat databázi.

\podsekce{Optimalizace databáze}
Co jsem použil pro optimalizaci

\podsekce{Dosažené výsledky}
Toto bude vždy na konci každého experimentu, grafy, časy, screeny, apod.

% Memcached
\sekce{Aplikační cache}
Tady se budeme snažit optimalizovat databázy.

\podsekce{Optimalizace aplikace pomocí Memcached}
Co jsem udělal s Memcached, jednotlivé vrstvy modelu, popis toho čeho chci dosáhnout.

\podsekce{Diagram tříd pro aplikaci s podporou Memcached}
Diagram tříd mé aplikace

\podsekce{Dosažené výsledky}
Toto bude vždy na konci každého experimentu, grafy, časy, screeny, apod.

% Reverse Proxy Cache
\sekce{Reverzní proxy cache}
Tady se budeme snažit použít reverzní proxy cache.

\podsekce{Nasazení a konfigurace NGINX s Memcached}
Jak jsem co dělal s NGINX, problémy, řešení, návrh architektury a jak to ovlivňuje aplikační vrstvu

\podsekce{Význam Ajax a webových služeb pro NGINX}
Ajax komunikuje s NGINX, proč a jak.

\podsekce{Dosažené výsledky}
Toto bude vždy na konci každého experimentu, grafy, časy, screeny, apod.


% Diskuze
\kapitola{Diskuze}
Diskutovaná řešení, jak je možné je kombinovat, apod.



% Zaver
\kapitola{Závěr}
Závěr ve smyslu nákladů a přínosů, kdy je lepší co. Uvidíme, napíšu jako poslední.






% \obrazek
% \vlozobrbox{test.jpg}{100mm}{100mm}
% \endobr{Popis obrázku}
% \begin{literatura}
% \citace{ucebnice}{Novák, 1991}{\autor{Novák, J.} a~kol.
% \nazev{Konstrukční vlastnosti ocelí třídy 18}. Praha:
% SNTL, 1991. 439~s. ISBN 80-8432-289-9.}
% \end{literatura}

\end{document}